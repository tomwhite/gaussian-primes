\documentclass[a4paper]{article}
\author{Tom~White}
\title{Gauss and Primes}
\begin{document}
\maketitle

\section{Introduction}

\subsection{Summary}

Gauss' empirical discovery of PNT.

Motivations for Gaussian primes - unique factorization. Applications.

End by asking: how are Gaussian primes distributed?

\subsection{The Conjecture}
In 1792, when he was only 15 years old\footnote{check, see Dunham pp78-79, Maor 184-185},
Carl Friedrich Gauss noticed a striking
connection between the distribution of the prime numbers and the logarithmic function.
Gauss had a table of prime numbers that had been compiled by Johann Lambert\footnote{ref?}, and
he was examining the number of prime numbers less than or equal to a given integer $x$.
It is now customary to denote this function as $\pi(x)$. Based purely on numerical evidence 
Gauss wrote
\begin{quote}
Primzahlen unter $a$ ($=\infty$) $a/\textrm{l}a$.\footnote{find a more original source than Maor}
\end{quote}
In today's notation we would write $\pi(x) \sim x/\log x$, that is, as $x$ tends to infinity
the ratio between $\pi(x)$ and $x/\log x$ tends to $1$. Gauss never proved his conjecture, indeed
it remained unproven for just over one hundred years until Hadamard and de la Vall\'ee-Poussin
independently proved in 1896 what became known as the Prime Number Theorem.

\subsection{Gaussian Primes}

[Gauss and Number Theory - King and Queen?]

While investigating the properties of biquadratic reciprocity (which we shan't go into)
Gauss introduced a generalisation of the integers, now known as \emph{Gauss\-ian integers}.\footnote{\cite{hardyandwright} p188 note 12.1}
These are the complex numbers $a + bi$, where $a$ and $b$ are both ordinary (rational) integers.
It turns out that these numbers behave in a manner which is analogous to rational integers.
For example, it is easy to check that the sums and products of Gaussian integers are themselves
Gaussian integers:
\begin{displaymath}
(a + bi) + (c + di) = (a + c) + (b + d)i
\end{displaymath}
\begin{displaymath}
(a + bi)(c + di) = (ac - bd) + (ad + bc)i
\end{displaymath}

Gauss proved that all Gaussian integers can be uniquely factored into prime Gaussian integers.
This is of course true of rational integers and Gauss was actually the first to
explicitly state the theorem for rational integers\footnote{DA, see ref in \cite{hardyandwright} p10, note 1.3},
athough the result was used by earlier mathematicians. But however familiar the property of
unique factorisation there are number systems that do not have this property. For instance, for
the numbers $a + b\sqrt{-5}$, where $a$ and $b$ are both rational integers
\begin{displaymath}
6 = 2\cdot 3 = (1 + \sqrt{-5})(1 - \sqrt{-5})
\end{displaymath}
and it can be shown that $2$, $3$, $1 + \sqrt{-5}$, and $1 - \sqrt{-5}$ are all prime.\footnote{do it!}

Having defined the Gaussian integers it is natural to ask: which Gaussian integers are prime?
How do the Gaussian primes relate to the rational primes? And in light of Gauss' discovery of
the distribution of the rational primes, how are the Gaussian primes distributed?

\section{What are Gaussian Primes?}

\subsection{Characterisation}
We first look at some definitions and simple results concerning Gaussian primes.
This treatment follows that in Hardy and Wright \cite{hardyandwright}.

A \emph{Gaussian integer} is a complex number $a + bi$, where $a$ and $b$ are both (rational) integers.
A \emph{unity} is a power of $i$, and we say a Gaussian integer $g$ is \emph{associated} with
$\epsilon g$ where $\epsilon$ is any unity. In other words the \emph{associates} of $g$ are
$g$, $ig$, $-g$, $-ig$.

The \emph{norm} of a Gaussian integer $a + bi$ is defined to be $N(a + bi) = a^{2} + b^{2}$.
The norm of a unity is always 1.

A \emph{Gaussian prime} is a Gaussian integer, not 0 or a unity, divisible only by numbers associated with itself or 1. 

Let's now consider the rational primes - are they prime when considered as Gaussian integers?
$2$ is not prime since it equals $(1 + i)(1 - i)$ - these factors are not associated with
itself ($2$) or 1. $3$ is still prime, this can be seen by trial division by all Gaussian integers
with norm less than or equal to 3 that are not 0 or a unity, that is: $1 + i$ (ignoring associates).
\begin{displaymath}
\frac{3}{1 + i} = \frac{3(1 - i)}{(1 + i)(1 - i)} = \frac{3(1 - i)}{1^{2} + 1^{2}}
= \frac{3}{2} - \frac{i}{2}
\end{displaymath}
and since $\frac{3}{2} - \frac{i}{2}$ is not a Gaussian integer $3$ is a Gaussian prime.
\footnote{See Bell in Vol 1 of World of Maths, on Gauss for why arithmetic is harder than algebra. This demonstration
that $3$ is still prime stirkes to the heart of it. More discussion please!}
Continuing in this way we can create a little table indicating when a rational prime $p$
is divisible by a Gaussian prime $a + bi$:

\begin{tabular}{rrr}
$p$ & $a$ & $b$ \\
\hline
2 & 1 & 1\\
3 \\
5 & 2 & 1\\
7 \\
11 \\
13 & 3 & 2\\
17 & 4 & 1\\
19 \\
23 \\
29 & 5 & 2\\
31 \\
37 & 6 & 1\\
41 & 5 & 4\\
43 \\
47 \\
\end{tabular}

There is a nice proof that rational primes of the form $4n + 1$ are \emph{never} Gaussian primes.
It starts from Wilson's test for primality: $p$ is prime if and only if $p$ divides $(p - 1) ! + 1$.\footnote{Reference proof}
Now $p = 4n + 1$, so it divides
\begin{eqnarray}
(4n)! + 1 & = & (1 \cdot 2 \cdot \ldots \cdot 2n \cdot (2n + 1) \cdot (2n + 2) \cdot \ldots \cdot 4n) + 1 \nonumber\\
          & \equiv & (1 \cdot 2 \cdot \ldots \cdot 2n \cdot (-2n) \cdot (-(2n - 1)) \cdot \ldots \cdot (-1)) + 1 \pmod p\nonumber\\
          & = & ((2n)!)^{2} + 1 \nonumber\\
          & = & ((2n)! + i)((2n)! - i) \nonumber
\end{eqnarray}
But $p$ divides neither factor so $p$ must have a Gaussian prime factor.
(In fact, the greatest common divisor of $(2n)! + i$ and $p$ is a prime factor of $p$.)

We can go further: if $a + bi$ is a (positive\footnote{define}) prime divisor of $p$ then so is $a - bi$:
\begin{displaymath}
\frac{p}{a \pm bi} = \frac{p(a \mp bi)}{(a \pm bi)(a \mp bi)} = \frac{p(a \mp bi)}{a^{2} + b^{2}}
= \frac{pa}{a^{2} + b^{2}} \mp \frac{pbi}{a^{2} + b^{2}}
\end{displaymath}

Since $a \neq b$, $a + bi$ and $a - bi$ are not associates, they are distinct primes, so their product
$(a + bi)(a - bi) = a^{2} + b^{2}$ divides $p$. But $p$ is a rational prime, so $p$ must \emph{equal}
$a^{2} + b^{2}$. This establishes...\footnote{Fermat}.

From the previous table it looks as if the rational primes of the form $4n + 3$
are also Gaussian primes. This is in fact true for all such primes and
we can prove this by supposing that $\pi \lambda = p$ where $\pi$ is a Gaussian prime. Then
\begin{displaymath}
N\pi N\lambda = p^{2}.
\end{displaymath}
Therefore either $N\lambda = 1$, in which case $\lambda$ is a unity and $\pi$ is an associate
of $p$, or
\begin{equation} \label{eq:0}
N\pi = a^{2} + b^{2} = p.
\end{equation}
Now a square is equal to 0 or 1 (mod 4), so (\ref{eq:0}) cannot be satisfied. Therefore the alternative
case must occur: $\pi$ is an associate of $p$, so $p$ is a Gaussian prime.

Incidently, equation (\ref{eq:0}) also shows that $1 + i$ (and its associates, of course) is prime.

\subsection{Proofs}

(Need Theorem 250 of \cite{hardyandwright}...)

Let $\pi = a + bi$ be a Gaussian prime. If
\begin{displaymath}
\pi \mid p, \pi \lambda = p,
\end{displaymath}
then 
\begin{displaymath}
N\pi N\lambda = p^{2}.
\end{displaymath}
Therefore either $N\lambda = 1$, in which case $\lambda$ is a unity (prove it!) and $\pi$ is an associate
of $p$, or
\begin{equation} \label{eq:1}
N\pi = a^{2} + b^{2} = p.
\end{equation}

We now look at different types of prime $p$.

\begin{enumerate}
\item If $p = 2$, then
\begin{displaymath}
p = 1^{2} + 1^{2}.
\end{displaymath}
So $1 + i $ and its associates are Gaussian primes.
\item If $p = 4n + 3$, then (\ref{eq:1}) cannot be satisfied since a square is equal to 0 or 1 (mod 4).
Hence ... (?)
\item If $p = 4n + 1$, then ...?
\end{enumerate}

Gaussian primes may be fully characterised by the following:
\begin{enumerate}
\item $1 + i$ and its associates ($-1 + i$, $-1 - i$, $1 - i$) 
\item the rational primes $4n + 3$ and their associates 
\item the factors $a + bi$ of the rational primes $4n + 1$ 
\end{enumerate}

This characterisation makes it easy to plot the distribution of Gaussian primes.
It is sufficient to consider only $a + bi$ where $a$ is positive, and $b$ lies between 0 and $a$.
In fact, $1 + i$ is the only such prime whose real part and imaginary part are equal.
For the rest there are two cases: 
\begin{enumerate}
\item If $b = 0$, $a + bi$ is prime if $a$ is a prime of the form $4n + 3$. 
\item If $b \neq 0$, $a + bi$ is prime if $a^{2} + b^{2}$ is prime. 
\end{enumerate}

These cases are easily checked using a list of (rational) prime numbers. 
My program draws the distribution of Gaussian primes bounded by a prescribed circle in the complex plane.
It uses a table of primes generated using the sieve of Eratosthenes. 

\section{The Prime Number Theorem}

\subsection{Summary}

Table of $x$, $\pi(x)$, etc.

Discussion of empirical results Gauss etc.

Description of the proof of PNT - historical. Number of different types of proofs.

Q: is $\pi_{4,1} > \pi_{4,3}$ always - current status. (Mention change of sign of $Li(x) - \pi(x)$)

Density result for Gaussian primes - derivation. Nice result linking primes, $e$ and $\pi$.

"Walking" from origin to infinity using the Gaussian primes as "stepping stones" and taking
steps of bounded length. See p34 Guy. Any progress on at least 4?

Visualisation discussion - Christmas tree, Op Art, scaling $r$ to even out the distribution - the
city skyline plot.

\subsection{Empirical observations}

\begin{tabular}{rrrrrr}
$x$       & $\pi(x)$    & $\lfloor x/\log x \rfloor$ & $\pi_{4,1}(x)$ & $\pi_{4,3}(x)$ & $\lfloor x/2\log x \rfloor$\\
\hline
$10^{1}$  & $4$         & $4$                       & $1$            & $2$            & $2$\\
$10^{2}$  & $25$        & $21$                      & $?$            & $?$\\
$10^{3}$  & $168$ \\
$10^{4}$  & $1,229$ \\
$10^{5}$  & $9,592$ \\
$10^{6}$  & $78,498$ \\
$10^{7}$  & $664,579$ \\
$10^{8}$  & $5,761,455$ \\
$10^{9}$  & $50,847,534$ \\
$10^{10}$ & $455,052,511$ & $434,294,481$\\
\end{tabular}

\subsection{The Density of Gaussian Primes}

We know from the Prime Number Theorem that $\pi(x) \sim x/\log x$. In other words the density of the
rational primes is $1/\log x$ for large $x$. Define $\pi_{d,a}(x)$ as the number of primes of the form
$dn + a$ less than or equal to $x$.\footnote{Ribenboim p 214} Then

\begin{displaymath}
\pi_{d,a}(x) \sim \frac{1}{\phi (d)}\frac{x}{\log x}
\end{displaymath}

Since $\phi (4)=2$ the density of primes of the form $4n + 1$ is equal to the density of those
of the form $4n + 3$, that is $1/2\log x$. This is intuitively plausible and consistent with the
empirical observations.\footnote{Previous table - link}

Now consider a circle of radius $r$ and the Gaussian primes in the sector $0 \leq \theta \leq \pi/2$.
For large $r$ there are $r/2\log r$ Gaussian primes on the real axis - those of the form $4n + 3$.
For those of the form $p = 4n + 1$, we know that $p = a^2 + b^2$ corresponds to the Gaussian
prime $a + bi$. For this to lie in the circle of radius $r$ requires that $\sqrt{a^2 + b^2} \leq r$, i.e.
$p \leq r^2$. So for large $r$ there are $r^2/2\log r^2 = r^2/4\log r$ such Gaussian primes in
the sector. So a long way from the origin the density is purely due to the primes of shape $4n + 1$,
and is

\begin{displaymath}
\frac{2}{\pi \log r}
\end{displaymath}



\begin{thebibliography}{9}
\bibitem{conwayandguy}Conway, John H. and Guy, Richard K., (1996),
\emph{The Book of Numbers}, Springer Verlag. 
\bibitem{hardyandwright}Hardy, G. H. and Wright, E. M. (1938),
\emph{The Theory of Numbers}, Oxford University Press, pp177-186, pp218-219.  
\bibitem{ribenboim}Ribenboim, Paulo, (1989),
\emph{The Book of Prime Number Records}, Springer Verlag. 
\bibitem{rose}Rose, H.E., (1988),
\emph{A Course in Number Theory}, Oxford University Press, p92.
\end{thebibliography}

\end{document}